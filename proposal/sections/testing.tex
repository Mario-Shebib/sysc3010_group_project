Our project concept has a number of modular components which can be tested
independently. The most critical components will be unit tested while less
critical components will only be tested at a module level due to time
constraints. We will also perform integration testing on the system as a whole.

This section describes the testing approach for each component of the project
and out planned approach to integration testing for the project as a whole.

\subsection{Door Nodes}

The door nodes contain hardware and software components which will need to be
tested individually and as a complete system.

The door node contains three main hardware components for which interface code
will need to be written and tested: the infrared temperature sensor, the NFC
security badge reader and the electronic lock actuator. To ease testing of the logic and
communications portions of the door node software we will write stubs which
emulate the functionality of each these hardware components. Our software will
be architected in such a way that the driver for any hardware component can be
easily replaced with its associated stub and that any combination of stubs and
real hardware can be used.

The hardware interface stubs will create log files that indicate how the
application software has made use of the interfaces provided by the hardware
interface code. They will also support the ability to emulate sensor data or
hardware triggered events based on test data provided as a text file. This
mechanism will be used to when hardware is not available, when testing with real
hardware is too cumbersome and to ensure that the the door node application code
reacts appropriately to invalid or missing data.

\subsubsection{Infrared Temperature Sensor}

\paragraph{Infrared Temperature Sensor Interface Code.}
We will write a minimal test application in order to test the infrared temperature sensor
hardware and interface code separately from the rest of the system. We will also
write unit tests with full line and branch coverage for the infrared temperature sensor
interface code. Some tests will be performed while monitoring the digital
interface of the infrared temperature sensor with a logic analyzer or oscilloscope in
order to verify that the interface code is meeting the timing requirements of
the digital sensor interface.

It will be important to test the hardware and software together to ensure that
we are able to make accurate temperature measurements. While we will be able to
test the sensor on ourselves to ensure that the results we receive are sane, we
need a test rig that is capable of creating arbitrary controllable temperatures
in order to properly test the sensor across the full range of temperatures that
we expect to encounter. Professionally this would be accomplished using a
blackbody calibration source, but these sources cost thousands of dollars. A
makeshift solution which will be sufficient within the scope of our project is
described in the below as part of our proposed approach to sensor calibration.

A stub will be developed that will allow emulated temperature data from a text
file to be fed to the door node application code.

\paragraph{Infrared Temperature Sensor Calibration.}
It is not trivial to achieve medical grade accuracy from an off the shelf
infrared temperature sensor module and some form of calibration will be required
in order to obtain useful readings. While a blackbody calibration source is
prohibitively expensive, for the scope of our project we can make due with a
makeshift calibration solution.

We will calibrate and test our infrared temperature sensors using a pot of water
heated to a target temperature. The infrared temperature sensor will be mounted at a
known height above the water, which will be heated on a stove to the desired
calibration temperature within the range of normal human body temperatures.
The water temperature will be measured with a standard mercury or digital
thermometer.

This solution will give us data for calibration that will be sufficiently
similar to temperature readings from human skin because water and human skin
have similar emissivities in the infrared spectrum. The emissivity of human skin
is roughly 0.98 \cite{Steketee_1973} and the emissivity of water is about 0.993
\cite{Buettner_1965}.

\subsubsection{NFC Reader}

We will develop a minimal test application that will allow us to test the NFC
reader hardware and interface code separately from the rest of the system. We
will also write unit tests with full line and branch coverage for the NFC reader
interface code. We will test our interface code using NFC cards that we have
prepared containing both invalid and valid data.

A stub will be developed that will allow emulated NFC security badge data from a text file
to be fed to the door node application code.

\subsubsection{Door Actuator}

A minimal test application will be developed to test the door actuator hardware
and interface code separately from the rest of the system.

A stub will be developed that will allow the door node application code to be
tested without the actuator hardware.

\subsection{Control Server}

In order to test the control server we will use the stubs described in section
\ref{subsec:testing-communication}. This will allow us to thoroughly
exercise the functions of the control server as it is being developed and before
the door node code is complete. This will also allow us to test that the
database and GUI portions of our system are able to support realistically large
numbers of door nodes despite the limited amount of door node hardware that we
have at our disposal.

\subsection{Communication}
\label{subsec:testing-communication}

We will test our communication interfaces by writing a stub application that
emulates one or more door nodes and a stub application that emulates the control
server.

These stubs will make use of our code for interacting with ThingSpeak and the
code that implements our communications protocol. This will allow us to test our
communications code by have the stubs for both sides communicate with each
other. We will also be able to test the control and door nodes of our system in
isolation by using the real code for one node and the stub for the other.

\subsection{Integration Testing}

Integration testing will be performed manually by running the control server and
door node code on real hardware communicating between nodes via ThingSpeak.
Integration testing may be augmented with additional emulated door nodes using
the door node stub.
