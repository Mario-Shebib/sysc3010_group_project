\subsubsection{Deployment Diagram}

\includegraphics[width=\textwidth]{figures/deployment.png}

\subsubsection{Communiction Between Nodes}
The nodes will be communicating with one another using Wifi. The door nodes do not communicate between themselves with their outputs informing the control node which in turn can send updates to the door nodes of changes that have occurred at other nodes.

\subsection{Door Nodes}
The door nodes are raspberry Pi that are set up in headless mode that will be the front line in the protection of the secured locations that use the product. They are connected to id cards, temperature probes and a door opener. They will not allow the doors to be opened if there is a full occupancy or if there is a temperature beyond the limit set by the control node. Furthermore, if that temperature occurs three times the control node will be given an option to set their id card to not allow them in until a quarantine period ends. An LED is set up with each of the door nodes that will be green if not yet at full capacity with the LED shining red if at full capacity to discourage people from attempting to enter the door.

\subsubsection{Hardware Block Diagram for Door Hardware}

\subsubsection{Usage Diagram for Door}

\includegraphics[width=\textwidth]{figures/door_interface.png}

\subsubsection{Sensors}
The temperature probe that will be used is central to the safety of the location and why the system exists. It will measure the temperature of the individual without physically touching them that will be communicated to the computer which has a program to determine if they have a fever which will not allow them to go through the door. If the temperature is within the acceptable range it will inform the door node to open and if it's below the acceptable range that will currently be determined to be the ambient temperature and keep the door closed as well. An NFC system will be used with id cards that will inform the Door node which can check if the individual should currently be in quarantine, if they don't have a valid card or if there have been enough other cards that have been swiped that the building is at maximum capacity.

\subsubsection{Actuator}
The door will open from an actuator that is attached to the door and will be activated upon permission being granted by the node. It will open with a pressure that is set as to not hurt individuals or individuals with children if the time they take to enter the door after it opens exceeds the default time. This is also to prevent any theoretical people on the other side of the door from being harmed by the door swinging open.

\subsection{Control Node}
The control node will be a Raspberry Pi connected to a monitor, keyboard and mouse. It will be where the operator of the security system can modify the acceptable range of temperatures and view the database for themselves as well as revoke access to those whose temperatures are at a high enough level that they will be asked to quarantine.

\subsubsection{Database}
The database will contain mostly the information about the individuals with valid ID cards and what the most recent result of the temperature probe was. Also, it will indicate whether those individuals are signed into the building or not, as well as how many individuals are currently signed into the building.

\subsubsection{GUI}
The Graphical User Interface of the Control Node will function primarily as a spreadsheet with an option to open up a pop-up to unilaterally turn on the red light for any of the door nodes as well as a pop up to change the acceptable temperature range. There will be a notification sidebar which will list the individuals who have had three temperatures above the temperature range limits in a row with an option to allow that ID to be used again with an O and to not allow them to be used for 14 days as a quarantine period with an X. In the spreadsheet, the operator can manually override these options if it is deemed necessary by them.
